\documentclass{beamer}

\usepackage{beamerthemesplit} % // Activate for custom appearance

\usepackage{xeCJK}
\usepackage{graphicx}
\usepackage{subcaption}
\usepackage{tikz}
\usepackage{pgfplots}
\usepackage{algpseudocode}
\usepackage{listings}

\lstset
{
  breaklines=true,
  breakatwhitespace=true,
}

\definecolor{nthu}{HTML}{7F1084}
\definecolor{secondary}{HTML}{910A17}
\definecolor{accent}{HTML}{410A91}
\definecolor{bg}{HTML}{171717}

\setbeamercolor{normal text}{fg=white, bg=bg}
\setbeamercolor{alerted text}{fg=secondary}
\setbeamercolor{example text}{fg=accent}
\setbeamercolor{background}{parent=normal text}
\setbeamercolor{background canvas}{parent=background}
\setbeamercolor{palette primary}{fg=white, bg=secondary}
\setbeamercolor{palette secondary}{use=structure, fg=white, bg=nthu}
\setbeamercolor{palette tertiary}{use=structure, fg=white, bg=accent}
\setbeamercolor{block title}{fg=white, bg=nthu!69!accent}
\setbeamercolor{block body}{fg=white, bg=accent!69!nthu}
\setbeamercolor{titlelike}{fg=white, bg=nthu}
\setbeamercolor{structure}{fg=nthu!64}
\usebeamercolor{structure}
%\usecolortheme[named=nthu]{structure}
\useinnertheme{rounded}
\useoutertheme{infolines}
\usefonttheme{serif}

\xeCJKsetup{CJKglue=\hspace{0pt plus .12 \baselineskip }}
\xeCJKsetup{RubberPunctSkip=false}
\xeCJKsetup{PunctStyle=plain}
\xeCJKsetup{CheckSingle=true}
\XeTeXlinebreaklocale "zh"
\XeTeXlinebreakskip = 0pt plus 2pt

\setCJKmainfont{Songti TC}
\setCJKsansfont{Apple LiGothic}
\setmonofont{Cascadia Code PL}

\title{排序演算法}
\subtitle{合併排序、快速排序……}
\author{nevikw39}
\institute{點石學園}
\date{\today}

\AtBeginSection{
	\frame
	{
%		\frametitle{Outline}
		\sectionpage
		\tableofcontents[sectionstyle=show/shaded, subsectionstyle=show/show/shaded, subsubsectionstyle=show/show/show/hide]
	}
}

\begin{document}

\frame{\titlepage}

\frame{\tableofcontents}

\section{Abstract}

\frame
{
	\frametitle{Introduction to Sorting}
	
	\begin{definition}[Sort]
		Rearrange elements in an array into a sort of order.
		
		\begin{itemize}
			\item Monotonicity
			\item Permutation of original array
		\end{itemize}
	\end{definition}
	\pause
	
	\begin{block}{Reason}
		\begin{itemize}
			\item Ranking
			\item Prerequisite of other algorithms such as binary search, greedy, ...
		\end{itemize}
	\end{block}
}

%\frame
%{
%	\frametitle{Na\"{i}ve Approach: Selection Sort}
%	
%	\pause
%	
%	\begin{algorithmic}
%		\Procedure{Selection Sort}{$\{a_0, a_1, ..., a_{n-1}\}$}
%            \For{$i\in[0, n-1)$}
%                \State$index\_of\_min\gets i$
%                \For{$j\in[i+1, n)$}
%                	\If{$a_{index\_of\_min} > a_j$}
%						\State$index\_of\_min\gets j$
%					\EndIf
%				\EndFor
%				\State\Call{swap}{$a_i, a_{index\_of\_min}$}
%            \EndFor
%            \State\Return$a$
%        \EndProcedure
%	\end{algorithmic}
%}

\subsection{Bubble Sort}

\frame
{
	\frametitle{Na\"{i}ve Approach: Bubble Sort}
	
	\begin{algorithmic}
		\Procedure{Bubble Sort}{$\{a_0, a_1, ..., a_{n-1}\}$}
            \For{$i\in[0, n-1)$}
                \For{$j\in[0, n-1-i)$}
                	\If{$a_j > a_{j + 1}$}
						\State\Call{swap}{$a_j, a_{j + 1}$}
					\EndIf
				\EndFor
            \EndFor
            \State\Return$a$
        \EndProcedure
	\end{algorithmic}
}

\frame{
	\frametitle{Example}
	
	\begin{center}
        \begin{tikzpicture}[scale=0.85, transform shape]
        	\only<1>{
        		\foreach \x/\val in {0/8,1/27,2/2021,3/110,4/20}{
        			\node[draw=gray,rectangle, fill=gray!92, minimum size =1cm] (c) at (\x,0) {\val};
        		}
        	}
        	\only<2>{
        		\foreach \x/\val/\col in {0/8/accent,1/27/accent,2/2021/gray,3/110/gray,4/20/gray}{
        			\node[draw=\col,rectangle, fill=\col!92, minimum size =1cm] (c) at (\x,0) {\val};
        		}
        	}
        	\only<3>{
        		\foreach \x/\val/\col in {0/8/gray,1/27/accent,2/2021/accent,3/110/gray,4/20/gray}{
        			\node[draw=\col,rectangle, fill=\col!92, minimum size =1cm] (c) at (\x,0) {\val};
        		}
        	}
        	\only<4>{
        		\foreach \x/\val/\col in {0/8/gray,1/27/gray,2/2021/secondary,3/110/secondary,4/20/gray}{
        			\node[draw=\col,rectangle, fill=\col!92, minimum size =1cm] (c) at (\x,0) {\val};
        		}
        	}
        	\only<5>{
        		\foreach \x/\val/\col in {0/8/gray,1/27/gray,2/110/accent,3/2021/accent,4/20/gray}{
        			\node[draw=\col,rectangle, fill=\col!92, minimum size =1cm] (c) at (\x,0) {\val};
        		}
        	}
        	\only<6>{
        		\foreach \x/\val/\col in {0/8/gray,1/27/gray,2/110/gray,3/2021/secondary,4/20/secondary}{
        			\node[draw=\col,rectangle, fill=\col!92, minimum size =1cm] (c) at (\x,0) {\val};
        		}
        	}
        	\only<7>{
        		\foreach \x/\val/\col in {0/8/gray,1/27/gray,2/110/gray,3/20/accent,4/2021/accent}{
        			\node[draw=\col,rectangle, fill=\col!92, minimum size =1cm] (c) at (\x,0) {\val};
        		}
        	}
        	\only<8>{
        		\foreach \x/\val/\col in {0/8/accent,1/27/accent,2/110/gray,3/20/gray,4/2021/gray}{
        			\node[draw=\col,rectangle, fill=\col!92, minimum size =1cm] (c) at (\x,0) {\val};
        		}
        	}
        	\only<9>{
        		\foreach \x/\val/\col in {0/8/gray,1/27/accent,2/110/accent,3/20/gray,4/2021/gray}{
        			\node[draw=\col,rectangle, fill=\col!92, minimum size =1cm] (c) at (\x,0) {\val};
        		}
        	}
        	\only<10>{
        		\foreach \x/\val/\col in {0/8/gray,1/27/gray,2/110/secondary,3/20/secondary,4/2021/gray}{
        			\node[draw=\col,rectangle, fill=\col!92, minimum size =1cm] (c) at (\x,0) {\val};
        		}
        	}
        	\only<11>{
        		\foreach \x/\val/\col in {0/8/gray,1/27/gray,2/20/accent,3/110/accent,4/2021/gray}{
        			\node[draw=\col,rectangle, fill=\col!92, minimum size =1cm] (c) at (\x,0) {\val};
        		}
        	}
        	\only<12>{
        		\foreach \x/\val/\col in {0/8/accent,1/27/accent,2/20/gray,3/110/gray,4/2021/gray}{
        			\node[draw=\col,rectangle, fill=\col!92, minimum size =1cm] (c) at (\x,0) {\val};
        		}
        	}
        	\only<13>{
        		\foreach \x/\val/\col in {0/8/gray,1/27/secondary,2/20/secondary,3/110/gray,4/2021/gray}{
        			\node[draw=\col,rectangle, fill=\col!92, minimum size =1cm] (c) at (\x,0) {\val};
        		}
        	}
        	\only<14>{
        		\foreach \x/\val/\col in {0/8/gray,1/20/accent,2/27/accent,3/110/gray,4/2021/gray}{
        			\node[draw=\col,rectangle, fill=\col!92, minimum size =1cm] (c) at (\x,0) {\val};
        		}
        	}
        	\only<15>{
        		\foreach \x/\val/\col in {0/8/accent,1/20/accent,2/27/gray,3/110/gray,4/2021/gray}{
        			\node[draw=\col,rectangle, fill=\col!92, minimum size =1cm] (c) at (\x,0) {\val};
        		}
        	}
        \end{tikzpicture}
	\end{center}
}

\section{Implementation}

\frame
{
	\frametitle{分(而)治(之)法 Divide \& Conquer}
	
	分治三部曲:
	\pause
	
	\begin{description}
		\item[Divide]Split original problem into several subproblems\pause
		\item[Conquer]Recur to each subproblem until it could be easily solved\pause
		\item[Combine]Merge the results of subproblems\pause
	\end{description}
	
	時間複雜度:Master Theorem
	\pause
	\[T(n) = aT(\frac{n}{b}) + O(n^d) = \left\{\begin{array}{ll}O(n^d), & d > \log_b a\\O(n^d\log n), & d =\log_b a\\O(n^{\log_b a}), & d < \log_b a\end{array}\right.\]
}

\subsection{Merge Sort}

\frame
{
	\frametitle{合併排序}
	
	\begin{description}
		\item<1->[Divide]Split array into $left$ and $right$ subarrays ($b = 2, O(1)$)
		\item<2->[Conquer]Sort two subarrays recursively ($a = 2$)
		\item<3->[Combine]Merge two sorted subarrays in $O(n)$
	\end{description}
}

\frame
{
	\frametitle{合併}
	
	Given two sorted subarrays $left, right$, how could we combine them to a sorted array efficiently??\pause
	
	Let \texttt{itr}, \texttt{jtr} point to the begin of $left, right$ respectively.\pause
	
	If $\texttt{itr} \neq \text{the end of }left \land \texttt{jtr} \neq \text{the end of }right$, then we choose the min one and forward the pointer.\pause
	
	Else if $\texttt{itr} \neq \text{the end of }left \lor \texttt{jtr} \neq \text{the end of }right$, then we choose the pointer and forward it.\pause
	
	Repeat until $\texttt{itr} = \text{the end of }left \land \texttt{jtr} = \text{the end of }right$.
}

\frame
{
	\frametitle{Pseudocode}
	
	\begin{algorithmic}
		\Procedure{Merge Sort}{$*begin, *end$}
			\pause
			\If{end - begin = 1}\Comment{0. Recursion boundary}
				\State\Return
			\EndIf
			\pause
            \State$mid\gets\frac{begin+end}{2}, left\gets[begin, mid), right\gets[mid, end)$\Comment{1. Divide}
            \pause
            \State\Call{Merge Sort}{$left.begin(), left.end()$}\Comment{2. Conquer}
            \State\Call{Merge Sort}{$right.begin(), right.end()$} 
            \pause
            \State$itr\gets left.begin(), jtr\gets right.begin()$
            \While{$begin\neq end$}\Comment{3. Combine}
            	\If{$itr\neq left.end()\land(jtr = right.end()\lor *itr<*jtr)$}
					\State$*begin\texttt{++}\gets*itr\texttt{++}$
				\Else
					\State$*begin\texttt{++}\gets*jtr\texttt{++}$
				\EndIf
            \EndWhile
        \EndProcedure
	\end{algorithmic}
}

\frame{
	\frametitle{Example}
	
	\begin{center}
        \begin{tikzpicture}[scale=0.85, transform shape]
        	\only<1>{
        		\foreach \x/\val in {0/8,1/27,2/2021,3/110,4/20}{
        			\node[draw=gray, rectangle, fill=gray!92, minimum size =1cm] (c) at (\x,0) {\val};
        		}
        	}
        	\only<2>{
        		\foreach \x/\val in {0/8,1/27,3/2021,4/110,5/20}{
        			\node[draw=gray, rectangle, fill=gray!92, minimum size =1cm] (c) at (\x,0) {\val};
        		}
        	}
        	\only<3>{
        		\foreach \x/\val in {0/8,2/27,4/2021,5/110,6/20}{
        			\node[draw=gray, rectangle, fill=gray!92, minimum size =1cm] (c) at (\x,0) {\val};
        		}
        	}
        	\only<4>{
        		\foreach \x/\val in {0/8,1/27,3/2021,4/110,5/20}{
        			\node[draw=gray, rectangle, fill=gray!92, minimum size =1cm] (c) at (\x,0) {\val};
        		}
        	}
        	\only<5>{
        		\foreach \x/\val in {0/8,1/27,4/2021,6/110,7/20}{
        			\node[draw=gray, rectangle, fill=gray!92, minimum size =1cm] (c) at (\x,0) {\val};
        		}
        	}
        	\only<6>{
        		\foreach \x/\val in {0/8,1/27,5/2021,8/110,10/20}{
        			\node[draw=gray, rectangle, fill=gray!92, minimum size =1cm] (c) at (\x,0) {\val};
        		}
        	}
        	\only<7>{
        		\foreach \x/\val in {0/8,1/27,4/2021,6/20,7/110}{
        			\node[draw=gray, rectangle, fill=gray!92, minimum size =1cm] (c) at (\x,0) {\val};
        		}
        	}
        	\only<8>{
        		\foreach \x/\val in {0/8,1/27,3/20,4/110,5/2021}{
        			\node[draw=gray, rectangle, fill=gray!92, minimum size =1cm] (c) at (\x,0) {\val};
        		}
        	}
        	\only<9->{
        		\foreach \x/\val in {0/8,1/20,2/27,3/110,4/2021}{
        			\node[draw=gray, rectangle, fill=gray!92, minimum size =1cm] (c) at (\x,0) {\val};
        		}
        	}
        \end{tikzpicture}
	\end{center}
}

\frame
{
	\frametitle{類題演練}
	
	\begin{itemize}
		\item AtCoder Beginner Contest 154 p. C - Distinct or Not
		\item LeetCode 0004. Median of Two Sorted Arrays (Hard!?)
	\end{itemize}
}

\frame
{
	\frametitle{Distinct or Not}
	\framesubtitle{AtCoder Beginner Contest 154 p. C}
	
	\begin{block}{Problem}
		Determine if each element in the array is unique.
	\end{block}
	\pause
	
	\begin{block}{Idea}
		Use nested loop??
	\end{block}
	\pause
	
	\begin{block}{Algorithm}
		In \textbf{combine} process of \textsc{Merge Sort}, we could check whether $\texttt{*itr} == \texttt{*jtr}$.
	\end{block}
}

\frame
{
	\frametitle{Median of Two Sorted Arrays}
	\framesubtitle{LeetCode 0004 (Hard!?)}
	
	\begin{definition}[Median]
		For an sorted array $a$ whose size is $n$, if $n$ is odd, then its median is $a_{n/ 2}$; otherwise, it is $\frac{a_{n/2} + a_{n/2 - 1}}{2}$
	\end{definition}
	\pause
	
	\begin{block}{Algorithm}
		Obviously, all we have to do is to perform \textbf{combine} process of \textsc{Merge Sort}.
	\end{block}
}

\subsubsection{Inversion}

\frame
{
	\frametitle{逆序對 (Inversion) 數量}
	
	\begin{definition}[Inversion]
		Given a sequence $S$. If $i < j \iff S_i > S_j$, then $(i, j)$ or $(S_i, S_j)$ is called a \textbf{inversion} of $S$.
	\end{definition}
	\pause
	
	\begin{block}{Idea}
		$\forall i \in [0, |S|)$, $\forall j \in [i, |S|)$, count whether $i < j \iff S_i > S_j$??
	\end{block}
	\pause
	
	\begin{block}{Algorithm}
		When \textbf{combining} two sorted subarrays, if $\texttt{*itr} > \texttt{*jtr}$, then a \textbf{inversion} exists, and \texttt{*(itr + 1)}, ... are also greater than \texttt{*jtr} with a smaller index.
	\end{block}
}

\frame
{
	\frametitle{類題演練}
	
	\begin{itemize}
		\item AtCoder Beginner Contest 190 p. F - Shift and Inversions
		\item UVa 10810: Ultra-QuickSort
		\item UVa 11858: Frosh Week
	\end{itemize}
}

\frame
{
	\frametitle{Shift and Inversions}
	\framesubtitle{AtCoder Beginner Contest 190 p. F}
	
	\begin{enumerate}
		\item<1->  $v = \{0, 1, 2, 3, 4\}$, there are $0$ inversions.
		\item<2->  $v = \{1, 2, 3, 4, 0\}$, there are $4$ inversions: $\textcolor{nthu!64}{(1, 0)}, \textcolor{secondary!64}{(2, 0), (3, 0), (4, 0)}$
		\item<3->  $v = \{2, 3, 4, 0, 1\}$, there are $6$ inversions: $\textcolor{accent!69}{(2, 0)}, (3, 0), (4, 0), \textcolor{nthu!64}{(2, 1)}, \textcolor{secondary!64}{(3, 1), (4, 1)}$
		\item<4-> $v = \{3, 4, 0, 1, 2\}$, there are $6$ inversions: $\textcolor{accent!69}{(3, 0), (3, 1)}, (4, 0), (4, 1), \textcolor{nthu!64}{(3, 2)}, \textcolor{secondary!64}{(4, 2)}$
		\item<5-> $v = \{4, 0, 1, 2, 3\}$, there are $4$ inversions: $(4, 0), (4, 1), (4, 2), \textcolor{secondary!64}{(4, 3)}$
	\end{enumerate}
	
	\begin{itemize}
    	\item<6->When removing $a$ from the front of the array, the number of inversion would decrease by $a$ for $a$ is the $a$-th smallest element in the array.
    	
    	\item<7->In like manner, when appending $a$ to the back of the array, the number would increase by $N - 1 - a$ because $a$ is less than $N - 1 - a$  elements in the array.
	\end{itemize}
}

\subsection{Quicksort}

\frame
{
	\frametitle{快速排序}
	
	\begin{description}
		\item<1->[Divide]Select a $pivot$ value and separate the array into $less$ partition and $greater$ partition ($b = 2, O(n)$)
		\item<2->[Conquer]Sort two partitions recursively ($a = 2$)
		\item<3->[Combine]Nothing to do, the array have been sorted. ($O(1)$)
	\end{description}
}

\frame
{
	\frametitle{原地分割 (In-place Division)}
	
	How could we separate the array into $less$ partition and $greater$ partition??\pause
	
	Suppose two partitions are $[begin, itr)$ and $[itr + 1, end)$.\pause
	
	Initially, $itr\gets begin$, $pivot\gets end-1$.\pause
	
	We traverse from $begin$ through $pivot$ with $jtr$.\pause
	
	If $*jtr < *pivot$, then we put it to $*itr$ and forward $itr$.\pause
	
	Finally, swap $*itr$ and $*pivot$.
}

\frame
{
	\frametitle{Pivot}
	
	The choice of $pivot$ plays a significant role when it comes to the efficiency of \textsc{Quicksort}.\pause
	
	In classical textbook ``Introduction to Algorithm'', C., S., L., R. chose the last element to be $pivot$.\pause
	
	Nevertheless, fixed $pivot$ may run into troubles because if unfortunately $pivot$ is minimum or maximum every time, the time complexity would decline to $O(n^2)$.\pause
	
	With an eye to avoiding this, we can pick $pivot$ randomly or use the median of $\{*begin, *(\frac{begin+end}{2}), *(end-1)\}$.
}

\frame
{
	\frametitle{Pseudocode}
	
	\begin{algorithmic}
		\Procedure{Quicksort}{$*begin, *end$}
			\pause
			\If{$end - begin \leq 1$}\Comment{0. Recursion boundary}
				\State\Return
			\EndIf
			\pause
            \State$itr\gets begin, pivot\gets end-1$\Comment{1. Divide}
            \For{$jtr\gets begin; jtr \neq pivot; jtr\texttt{++}$}
                \If{$*jtr < *pivot$}
                    \State\Call{swap}{$*itr\texttt{++}, *jtr$}
                \EndIf
            \EndFor
            \State\Call{swap}{$*itr, *pivot$}
            \pause
            \State\Call{Quicksort}{$begin, itr$}\Comment{2. Conquer}
            \State\Call{Quicksort}{$itr+1, end$}
            \pause
            \State\Comment{3. Combine}
        \EndProcedure
	\end{algorithmic}
}

\frame{
	\frametitle{Example}
	
	\begin{center}
        \begin{tikzpicture}[scale=0.85, transform shape]
        	\only<1>{
        		\foreach \x/\val/\col in {0/8/gray,1/27/gray,2/2021/gray,3/110/gray,4/20/nthu}{
        			\node[draw=\col,rectangle, fill=\col!92, minimum size =1cm] (c) at (\x,0) {\val};
        		}
        	}
        	\only<2>{
        		\foreach \x/\val/\col in {0/8/gray,2/20/nthu,4/27/gray,5/2021/gray,6/110/nthu}{
        			\node[draw=\col,rectangle, fill=\col!92, minimum size =1cm] (c) at (\x,0) {\val};
        		}
        	}
        	\only<3>{
        		\foreach \x/\val/\col in {0/8/gray,3/20/nthu,6/27/gray,8/110/nthu,10/2021/gray}{
        			\node[draw=\col,rectangle, fill=\col!92, minimum size =1cm] (c) at (\x,0) {\val};
        		}
        	}
        	\only<4>{
        		\foreach \x/\val/\col in {0/8/gray,2/20/nthu,4/27/gray,5/110/gray,6/2021/gray}{
        			\node[draw=\col,rectangle, fill=\col!92, minimum size =1cm] (c) at (\x,0) {\val};
        		}
        	}
        	\only<5>{
        		\foreach \x/\val/\col in {0/8/gray,1/20/gray,2/27/gray,3/110/gray,4/2021/gray}{
        			\node[draw=\col,rectangle, fill=\col!92, minimum size =1cm] (c) at (\x,0) {\val};
        		}
        	}
        \end{tikzpicture}
	\end{center}
}

\subsubsection{Quickselect}

\frame
{
	\frametitle{快速選擇 Quickselect}
	
	\begin{block}{Problem}
		Find the $k$-th element in the array.
	\end{block}
	\pause
	
	\begin{block}{Idea}
		Sort the array??
	\end{block}
	\pause
	
	\begin{block}{Algorithm}
		When \textbf{dividing} array into $less$ and $greater$ parts in \textsc{Quicksort}, we could know the size of them. If $k < |less|$, we only need to recur to less part and greater one could be ignored; vice versa.
		
		Note that if we recur to greater part, we should update $k$ to $k - |less| - 1$ because the $k$-th element in the original array is $k - |less| - 1$-th element in the greater part.
	\end{block}
}

\frame
{
	\frametitle{類題演練}
	
	\begin{itemize}
		\item LeetCode 0215. Kth Largest Element in an Array (Medium)
		\item AtCoder Beginner Contest 161 p. C - Popular Vote
	\end{itemize}
}

\frame
{
	\frametitle{Kth Largest Element in an Array}
	\framesubtitle{LeetCode 0215 (Medium)}
	
	非常單純的模板題,我們以此來示範 $pivot$ 之選擇。
	
	\begin{description}
		\item[Always last] 40ms, beats $17.10\%$ submissions
		\item[Randomly] 8ms, faster than $75.44\%$ codes
	\end{description}
}

\section{Easy-to-use built-in functions}

\frame
{
	\frametitle{\ttfamily std::merge()}
	
	\small
	\lstinputlisting[language=C++, firstline=8, lastline=21]{10_std_func.cpp}
}

\frame
{
	\frametitle{\ttfamily std::partition()}
	
	\small
	\lstinputlisting[language=C++, firstline=23, lastline=42]{10_std_func.cpp}
}

\frame
{
	\frametitle{\ttfamily std::nth\_element()}
	
	\lstinputlisting[language=C++, firstline=74, lastline=75]{10_std_func.cpp}
}

\subsubsection{std::sort()}

\frame
{
	\frametitle{\ttfamily std::sort()}
	
	\lstinputlisting[language=C++, firstline=83, lastline=85]{10_std_func.cpp}
}

\frame
{
	\frametitle{\texttt{std::sort()} with \texttt{std::string} and other...}
	
	\lstinputlisting[language=C++, firstline=87, lastline=90]{10_std_func.cpp}
}

\begin{frame}[fragile]
	\frametitle{\texttt{std::sort()} using custom comparator}
	
	\small
	\lstinputlisting[language=C++, firstline=41, lastline=47]{10_std_func.cpp}
	\begin{lstlisting}[language=C++]
// ...
    \end{lstlisting}
	\lstinputlisting[language=C++, firstline=97, lastline=99]{10_std_func.cpp}
\end{frame}

\section{Appendix}

\frame
{
	\frametitle{Stable sorts vs. Unstable sorts}
	
	\begin{definition}[Stability]
		For non-distinct elements, whether they are outputted in the same order they inputted.
	\end{definition}
}

\frame
{
	\frametitle{Comparison sorts vs. Distribution sorts}
	
	\textsc{Bubble sort}, \textsc{merge sort}, \textsc{Quicksort} and ... are called \textit{Comparison sorts} because all of them are based on a binary boolean relation (oft $ < $) and aimed at swapping all inversions.
	
	On the other hand, there are other means.
}

\frame
{
	\frametitle{Age sort}
	\framesubtitle{UVa 11462}
	
	There are up to $2 \times 10^6$ people aging between $[1, 100]$ in the country. Our goal is to sort them by their age.
}

\frame
{
	\frametitle{Counting sort}
	
	If the \textit{range} of input is small enough, then we could record the occurrence of for all element in the \textit{range}.
}

\end{document}
