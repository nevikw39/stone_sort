\documentclass{beamer}

\usepackage{beamerthemesplit} % // Activate for custom appearance

\usepackage{xeCJK}
\usepackage{graphicx}
\usepackage{subcaption}
\usepackage{tikz}
\usepackage{pgfplots}
\usepackage{algpseudocode}

\definecolor{nthu}{HTML}{7F1084}
\definecolor{secondary}{HTML}{910A17}
\definecolor{accent}{HTML}{410A91}

\usetheme{Warsaw}
\usecolortheme[named=nthu]{structure}
\useinnertheme{rounded}
\useoutertheme{infolines}
\usefonttheme{serif}

\xeCJKsetup{CJKglue=\hspace{0pt plus .12 \baselineskip }}
\xeCJKsetup{RubberPunctSkip=false}
\xeCJKsetup{PunctStyle=plain}
\xeCJKsetup{CheckSingle=true}
\XeTeXlinebreaklocale "zh"
\XeTeXlinebreakskip = 0pt plus 2pt

\setCJKmainfont{Songti TC}
\setCJKsansfont{Apple LiGothic}
\setmonofont{Cascadia Code PL}

\title{排序演算法}
\subtitle{合併排序、快速排序……}
\author{nevikw39}
\institute{點石學園}
\date{\today}

\AtBeginSection{
	\frame
	{
%		\frametitle{Outline}
		\sectionpage
		\tableofcontents[currentsection, hideothersubsections]
	}
}

\begin{document}

\frame{\titlepage}

\frame{\tableofcontents}

\section{序}

\frame
{
	\frametitle{Introduction to Sorting}
	
	\begin{definition}
		Rearrange elements in an array into a sort of order.
		
		\begin{itemize}
			\item Monotonicity
			\item Permutation of original array
		\end{itemize}
	\end{definition}
	
	\begin{block}{Reason}
		\begin{itemize}
			\item Ranking
			\item Prerequisite of other algorithms such as binary search, greedy, ...
		\end{itemize}
	\end{block}
}

\frame
{
	\frametitle{Na\"{i}ve Approach: Selection Sort}
	
	\pause
	
	\begin{algorithmic}
		\Procedure{Selection Sort}{$\{a_0, a_1, ..., a_{n-1}\}$}
            \For{$i\in[0, n-1)$}
                \State$index\_of\_min\gets i$
                \For{$j\in[i+1, n)$}
                	\If{$a_{index\_of\_min} > a_j$}
						\State$index\_of\_min\gets j$
					\EndIf
				\EndFor
				\State\Call{swap}{$a_i, a_{index\_of\_min}$}
            \EndFor
            \State\Return$a$
        \EndProcedure
	\end{algorithmic}
}

\subsection{Bubble Sort}

\frame
{
	\frametitle{Slightly Improve: Bubble Sort}
	
	\begin{algorithmic}
		\Procedure{Bubble Sort}{$\{a_0, a_1, ..., a_{n-1}\}$}
            \For{$i\in[0, n-1)$}
                \For{$j\in[0, n-1-i)$}
                	\If{$a_j > a_{j + 1}$}
						\State\Call{swap}{$a_j, a_{j + 1}$}
					\EndIf
				\EndFor
            \EndFor
            \State\Return$a$
        \EndProcedure
	\end{algorithmic}
}

\frame{
	\frametitle{Bubble Sort Example}
	
	\begin{center}
        \begin{tikzpicture}[scale=0.85, transform shape]
        	\only<1>{
        		\foreach \x/\val in {0/8,1/27,2/2021,3/110,4/20}{
        			\node[draw,rectangle, fill=gray!10, minimum size =1cm] (c) at (\x,0) {\val};
        		}
        	}
        	\only<2>{
        		\foreach \x/\val/\col in {0/8/accent,1/27/accent,2/2021/black,3/110/black,4/20/black}{
        			\node[draw=\col,rectangle, fill=\col!10, minimum size =1cm] (c) at (\x,0) {\val};
        		}
        	}
        	\only<3>{
        		\foreach \x/\val/\col in {0/8/black,1/27/accent,2/2021/accent,3/110/black,4/20/black}{
        			\node[draw=\col,rectangle, fill=\col!10, minimum size =1cm] (c) at (\x,0) {\val};
        		}
        	}
        	\only<4>{
        		\foreach \x/\val/\col in {0/8/black,1/27/black,2/2021/secondary,3/110/secondary,4/20/black}{
        			\node[draw=\col,rectangle, fill=\col!10, minimum size =1cm] (c) at (\x,0) {\val};
        		}
        	}
        	\only<5>{
        		\foreach \x/\val/\col in {0/8/black,1/27/black,2/110/accent,3/2021/accent,4/20/black}{
        			\node[draw=\col,rectangle, fill=\col!10, minimum size =1cm] (c) at (\x,0) {\val};
        		}
        	}
        	\only<6>{
        		\foreach \x/\val/\col in {0/8/black,1/27/black,2/110/black,3/2021/secondary,4/20/secondary}{
        			\node[draw=\col,rectangle, fill=\col!10, minimum size =1cm] (c) at (\x,0) {\val};
        		}
        	}
        	\only<7>{
        		\foreach \x/\val/\col in {0/8/black,1/27/black,2/110/black,3/20/accent,4/2021/accent}{
        			\node[draw=\col,rectangle, fill=\col!10, minimum size =1cm] (c) at (\x,0) {\val};
        		}
        	}
        	\only<8>{
        		\foreach \x/\val/\col in {0/8/accent,1/27/accent,2/110/black,3/20/black,4/2021/black}{
        			\node[draw=\col,rectangle, fill=\col!10, minimum size =1cm] (c) at (\x,0) {\val};
        		}
        	}
        	\only<9>{
        		\foreach \x/\val/\col in {0/8/black,1/27/accent,2/110/accent,3/20/black,4/2021/black}{
        			\node[draw=\col,rectangle, fill=\col!10, minimum size =1cm] (c) at (\x,0) {\val};
        		}
        	}
        	\only<10>{
        		\foreach \x/\val/\col in {0/8/black,1/27/black,2/110/secondary,3/20/secondary,4/2021/black}{
        			\node[draw=\col,rectangle, fill=\col!10, minimum size =1cm] (c) at (\x,0) {\val};
        		}
        	}
        	\only<11>{
        		\foreach \x/\val/\col in {0/8/black,1/27/black,2/20/accent,3/110/accent,4/2021/black}{
        			\node[draw=\col,rectangle, fill=\col!10, minimum size =1cm] (c) at (\x,0) {\val};
        		}
        	}
        	\only<12>{
        		\foreach \x/\val/\col in {0/8/black,1/27/black,2/20/black,3/110/accent,4/2021/accent}{
        			\node[draw=\col,rectangle, fill=\col!10, minimum size =1cm] (c) at (\x,0) {\val};
        		}
        	}
        	\only<13>{
        		\foreach \x/\val/\col in {0/8/accent,1/27/accent,2/20/black,3/110/black,4/2021/black}{
        			\node[draw=\col,rectangle, fill=\col!10, minimum size =1cm] (c) at (\x,0) {\val};
        		}
        	}
        	\only<14>{
        		\foreach \x/\val/\col in {0/8/black,1/27/secondary,2/20/secondary,3/110/black,4/2021/black}{
        			\node[draw=\col,rectangle, fill=\col!10, minimum size =1cm] (c) at (\x,0) {\val};
        		}
        	}
        	\only<15>{
        		\foreach \x/\val/\col in {0/8/black,1/20/accent,2/27/accent,3/110/black,4/2021/black}{
        			\node[draw=\col,rectangle, fill=\col!10, minimum size =1cm] (c) at (\x,0) {\val};
        		}
        	}
        	\only<16>{
        		\foreach \x/\val/\col in {0/8/black,1/20/black,2/27/accent,3/110/accent,4/2021/black}{
        			\node[draw=\col,rectangle, fill=\col!10, minimum size =1cm] (c) at (\x,0) {\val};
        		}
        	}
        	\only<17>{
        		\foreach \x/\val/\col in {0/8/black,1/20/black,2/27/black,3/110/accent,4/2021/accent}{
        			\node[draw=\col,rectangle, fill=\col!10, minimum size =1cm] (c) at (\x,0) {\val};
        		}
        	}
        	\only<18>{
        		\foreach \x/\val/\col in {0/8/accent,1/20/accent,2/27/black,3/110/black,4/2021/black}{
        			\node[draw=\col,rectangle, fill=\col!10, minimum size =1cm] (c) at (\x,0) {\val};
        		}
        	}
        	\only<19>{
        		\foreach \x/\val/\col in {0/8/black,1/20/accent,2/27/accent,3/110/black,4/2021/black}{
        			\node[draw=\col,rectangle, fill=\col!10, minimum size =1cm] (c) at (\x,0) {\val};
        		}
        	}
        	\only<20>{
        		\foreach \x/\val/\col in {0/8/black,1/20/black,2/27/accent,3/110/accent,4/2021/black}{
        			\node[draw=\col,rectangle, fill=\col!10, minimum size =1cm] (c) at (\x,0) {\val};
        		}
        	}
        	\only<21->{
        		\foreach \x/\val/\col in {0/8/black,1/20/black,2/27/black,3/110/accent,4/2021/accent}{
        			\node[draw=\col,rectangle, fill=\col!10, minimum size =1cm] (c) at (\x,0) {\val};
        		}
        	}
        \end{tikzpicture}
	\end{center}
}

\section{原理}

\frame
{
	\frametitle{分(而)治(之)法 Divide \& Conquer}
	
	分治三部曲:
	\pause
	
	\begin{enumerate}
		\item Divide\pause
		\item Conquer\\frequently using recursion\pause
		\item Combine\pause
	\end{enumerate}
	
	時間複雜度:Master Theorem
	\pause
	\[T(n) = aT(\frac{n}{b}) + O(n^d) = \left\{\begin{array}{ll}O(n^d), & d > \log_b a\\O(n^d\log n), & d =\log_b a\\O(n^{\log_b a}), & d < \log_b a\end{array}\right.\]
}

\subsection{Merge Sort}

\frame
{
	\frametitle{合併排序}
	
	\begin{enumerate}
		\item<2-> Divide\\Split array into left and right subarrays ($b = 2$)
		\item<3-> Conquer\\Sort two subarrays recursively ($a = 2$)
		\item<4-> Combine\\Merge two sorted subarrays in $O(n)$
	\end{enumerate}
}

\frame
{
	\frametitle{合併}
	
	Given $a, b$ are two sorted subarrays, how could we combine them to a sorted array efficiently??\pause
	
	Let \texttt{itr}, \texttt{jtr} point to the begin of $a, b$ respectively.\pause
	
	If $\texttt{itr} \neq \text{the end of }a \land \texttt{jtr} \neq \text{the end of }b$, then we choose the min one and increase the pointer.\pause
	
	Else if $\texttt{itr} \neq \text{the end of }a \lor \texttt{jtr} \neq \text{the end of }b$, then we choose the pointer and increase it.\pause
	
	Repeat until $\texttt{itr} = \text{the end of }a \lor \texttt{jtr} = \text{the end of }b$.
}

\frame
{
	\frametitle{虛擬碼}
	
	\begin{algorithmic}
		\Procedure{Merge Sort}{$*begin, *end\}$}
			\pause
			\If{end - begin = 1}\Comment{0. Recursion boundary}
				\State\Return
			\EndIf
			\pause
            \State$mid\gets\frac{begin+end}{2}, a\gets[begin, mid), b\gets[mid, end)$\Comment{1. Divide}
            \pause
            \State\Call{Merge Sort}{$a.begin(), a.end()$}\Comment{2. Conquer}
            \State\Call{Merge Sort}{$b.begin(), b.end()$}, 
            \pause
            \State$itr\gets a.begin(), jtr\gets b.begin()$
            \While{$begin\neq end$}\Comment{3. Combine}
            	\If{$itr\neq a.end()\land(jtr = b.end()\lor *itr<*jtr)$}
					\State$*begin\texttt{++}\gets*itr\texttt{++}$
				\Else
					\State$*begin\texttt{++}\gets*jtr\texttt{++}$
				\EndIf
            \EndWhile
        \EndProcedure
	\end{algorithmic}
}

\frame{
	\frametitle{Example}
	
	\begin{center}
        \begin{tikzpicture}[scale=0.85, transform shape]
        	\only<1>{
        		\foreach \x/\val in {0/8,1/27,2/2021,3/110,4/20}{
        			\node[draw,rectangle, fill=gray!10, minimum size =1cm] (c) at (\x,0) {\val};
        		}
        	}
        	\only<2>{
        		\foreach \x/\val in {0/8,1/27,3/2021,4/110,5/20}{
        			\node[draw,rectangle, fill=gray!10, minimum size =1cm] (c) at (\x,0) {\val};
        		}
        	}
        	\only<3>{
        		\foreach \x/\val in {0/8,1/27,4/2021,6/110,7/20}{
        			\node[draw,rectangle, fill=gray!10, minimum size =1cm] (c) at (\x,0) {\val};
        		}
        	}
        	\only<4>{
        		\foreach \x/\val in {0/8,1/27,5/2021,8/110,10/20}{
        			\node[draw,rectangle, fill=gray!10, minimum size =1cm] (c) at (\x,0) {\val};
        		}
        	}
        	\only<5>{
        		\foreach \x/\val in {0/8,1/27,4/2021,6/20,7/110}{
        			\node[draw,rectangle, fill=gray!10, minimum size =1cm] (c) at (\x,0) {\val};
        		}
        	}
        	\only<6>{
        		\foreach \x/\val in {0/8,1/27,3/20,4/110,5/2021}{
        			\node[draw,rectangle, fill=gray!10, minimum size =1cm] (c) at (\x,0) {\val};
        		}
        	}
        	\only<7->{
        		\foreach \x/\val in {0/8,1/20,2/27,3/110,4/2021}{
        			\node[draw,rectangle, fill=gray!10, minimum size =1cm] (c) at (\x,0) {\val};
        		}
        	}
        \end{tikzpicture}
	\end{center}
}

\subsection{Quick Sort}

\section{實務}

\end{document}
